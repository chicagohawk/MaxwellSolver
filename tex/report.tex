\documentclass[a4paper,onecolumn]{article}
\usepackage{amsmath, amsthm, graphicx, amssymb, wrapfig, fullpage, subfigure, array}
\usepackage[font=sl, labelfont={sf}, margin=1cm]{caption}
\DeclareMathOperator{\e}{e}
\begin{document}
\setcounter{page}{1}

\noindent This is a description about the content in the tar file.
\section{2D Problem Description}
For 2D TM wave, the equation for $\vec{E}$ is reduced to a scalar equation
\begin{equation}
    \nabla\cdot \left( \frac{1}{\mu_r} \nabla E_z \right) + k^2 \epsilon E_z = 0
    \label{eq1}
\end{equation}
with appropriate boundary condition. Similar equation for $H_z$ holds for TE wave. We focus on \eqref{eq1}\\
\subsection{Homogeneous Dirichlet boundary}
For homogeneous Dirichlet boundary condition, this equation describes a resonance problem.
Finite element discretization of \eqref{eq1} yields an algebraic general eigenvalue problem.
\begin{equation}
    Au = k^2 Bu
\end{equation}
from which we solve for the resonance modes $u$'s and resonance frequencies $k^2$.
\subsection{Inhomogeneous Dirichlet boundary}
The discretized equation is
\begin{equation}
    Cu = f
\end{equation}
We solve for the unique solution $u$.
The codes can also solve for Neumann and Robin boundaries, though we do not show examples in the tar file.
\section{Folder Circle}
A circular geometry is chosen, with homogeneous Dirichlet boundary condition. This folder gives the mesh, resonant modes, and
eigenvalues ($k^2$). For simplicity and verification both $\epsilon_r$ and $\mu_r$ are chosen to be constant. Notice all the numbers
are non-dimensionalized.
\section{Folder Complex}
A more complex geometry.
\subsection{Folder Geometry}
As shown in antena\_bnd.png, this geometry has interior (red) and exterior (yellow) boundaries. It has 8704 elements.
The relative magnetic permeability is a spatial varying field. The relative permettivity is assume a constant field, though
a varying permettivity is also straightforward.
\subsection{Folder Resonance}
When a homogeneous Dirichlet boundary condition is applied to both interior and exterior boundaries, we solve for an eigenvalue problem.
The pictures in each folder shows the corresponding eigen-mode: one for magnitude, another for phase angle (normalized by $\pi$).
It is interesting that the first 6 eigen modes explores the vertical bars one by one. For example, the 1st and 2nd modes are both concentrated
in the center vertical bar, while the 1st mode vibrates at the same phase angle up and down, and the 2nd mode vibrates at an opposite
phase angle up and down.\\
We also give an example of higher modes, mode 35.\\
The eigenvalues, i.e. $k^2$ for the first 100 modes are plotted in eigmode.png.
\subsection{Folder Driven}
In this example the interior boundary is still homogeneous Dirichlet, but the exterior boundary is set to be $1$, i.e. uniformly
driven by an exterior wave. The $k^2$ just corresponds to the exterior driven frequency. How would the electric field look like as we gradually
increase the driven frequency? This is shown in the movie ($k^2$from 0 to 2000), only the magnitude is shown here.\\
When $k^2=0$ it is a electrostatic problem, which corresponds to the very start of the movie.\\

\end{document}









