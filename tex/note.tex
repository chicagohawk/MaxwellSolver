\documentclass[a4paper,onecolumn]{article}
\usepackage{amsmath, amsthm, graphicx, amssymb, wrapfig, fullpage, subfigure, array}
\usepackage[font=sl, labelfont={sf}, margin=1cm]{caption}
\DeclareMathOperator{\e}{e}
\begin{document}
\setcounter{page}{1}

\section{Problem Statement}
Spaces
\begin{equation*}\begin{split}
    X^D &= \{v\in H^1(\Omega) \;\bigl.\bigr|\; v|_{\partial \Omega^D} = u^D\}\\
    X &= \{v\in H^1(\Omega) \;\bigl.\bigr|\; v|_{\partial \Omega^D} = 0\}
\end{split}\end{equation*}
Notation
\begin{equation*}\begin{split}
    A_e^R &= \partial\Omega^R \cap A_e\\
    A_e^N &= \partial\Omega^N \cap A_e
\end{split}\end{equation*}
Solution space
$$
    u\in X^D
$$
Test space
$$
    v\in X
$$
Equation
\begin{equation*}
    \nabla \cdot (a \nabla u) + k^2 u = s
\end{equation*}
with 3 kinds of boundary conditions (Dirichlet, Neumann, Robin)
\begin{equation*}\begin{split}
    u\bigl.\bigr|_{\partial \Omega^D} &= u^D\\
    \hat{n} \cdot \nabla u \bigl.\bigr|_{\partial \Omega^N}&= f\\
    \Bigr(\gamma u + \frac{\partial u}{\partial \hat{n}} \Bigr)\Bigl.\Bigr|_{\partial \Omega^R}&= g
\end{split}\end{equation*}
Weak form
\begin{equation*}
    -\int_\Omega a \nabla u \cdot \nabla v + \int_\Omega k^2 u v 
    -\int_{\partial \Omega^R} \gamma a u v =  \int_\Omega s v
    - \int_{\partial \Omega^R} ag v
    - \int_{\partial \Omega^N} af v
\end{equation*}
Proceed as $u^D = 0$, i.e. homogeneous Dirichlet, and use explicit elimination in the last step.\\
Assume $\,f, \,g,\, u^D, \,s, \,a,\,\gamma$ elemental / piecewise constant.\\

\noindent Elemental contribution ($A_e$)
\begin{itemize}
    \item Bilinear matrix
    \begin{equation*}
        -\int_{A_e}  a \Bigl( J^{-T} \nabla_L L_i \Bigr)^T \cdot \Bigl(J^{-T}\nabla_L L_j\Bigr)
        + \int_{A_e} k^2 L_i L_j - \int_{A_e^R} \gamma a L_i L_j
    \end{equation*}
    \item Load vector
    \begin{equation*}
        \int_{A_e} s L_i - \int_{A_e^R} a g L_i - \int_{A_e^N} a f L_i
    \end{equation*}
\end{itemize}
In which,
\begin{equation}
(J^T)^{-1}=\frac{1}{2A_e}
\begin{pmatrix}
y_2-y_3 & -y_1+y_3\\
\\
-x_2+x_3 & x_1-x_3
\end{pmatrix}
\end{equation}
\begin{equation}
\nabla_L \vec{L}=
\begin{pmatrix}
1 & 0 & -1\\
\\
0 & 1 & -1
\end{pmatrix}
\end{equation}
\begin{equation}\begin{split}
\int_{A_e}\! L_1^a\, L_2^b\, L_3^c&=a!\,b!\,c!\,\frac{2A_e}{(a+b+c+2)\,!}
\end{split}\end{equation}
\noindent For a boundary edge of length $L$, let $i,\,j$ denote the nodal basis index on the edge.
\begin{equation*}\begin{split}
    \int_e L_i L_j &= \frac{L}{6}(1+\delta_{ij})\\
    \int_e L_i &= \frac{L}{2}
\end{split}\end{equation*}
Explicit elimination. Denote $\{I^D\}$ as the nodal indices on $\partial \Omega^D$
\begin{itemize}
    \item Remove $\{I^D\}$ rows of the equation
    \item Load vector
          $$
              F = F - M{\{:,I^D\}} \cdot u^{D}
          $$
    \item Remove $\{I^D\}$ columns of the bilinear matrix
\end{itemize}
Finally, solve
$$
    Mu = F
$$
For the general eigenvalue problem,
$$
    A_1 u = \lambda A_2 u
$$
we have
$$
    A u \;\rightarrow\; A(\neg I^D, \neg I^D) u(\neg I^D) + A(\neg I^D, I^D) u(I^D) \equiv \bar{A}\bar{u}+f
$$
Thus the general eigenvalue problem reduces to
$$
    \left( \bar{A}_1 - \bar{A}_2 \right) \bar{u} = f_2 - f_1
$$
The solution can be unique for non-homogeneous Dirichlet boundary conditions.

\end{document}








